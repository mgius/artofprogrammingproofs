\documentclass[10pt]{article}
\usepackage{amsmath}

\begin{document}
\author{Mark Gius}
\title{Product Divisors}
\maketitle

\section{Introduction}

While reading Section 1.3.2, a sample MIX program designed to determine the first 100 primes briefly mentioned an optimization that centered around the idea that if you have checked every factor of $X$ up to $ \sqrt{X} $ then you can stop and declare the number prime. The root of this optimization is a claim that for each pair of factors of a number, one must be less than $ \sqrt{X} $ and the other must be greater.  I wanted to see why.

\section{Proof/Analysis}

Let's start with the claim.

\[
\textrm{if } x = yz \textrm{ then } y > \sqrt{x} \textrm{ and } z < \sqrt{x} \textrm{ or } y = z
\]

My first thought was to prove that every possibility other than the one suggested above was invalid.  This led to examples of why $y$ and $z$ cannot both be greater or less than $\sqrt{x}$.

\begin{align*}
\textrm{Assume } y,z &> \sqrt{x} \\
yz &> x \\
yz &= x \\
\textbf{Contradiction} \\
\newline
\textrm{Assume } y,z &< \sqrt{x} \\
yz &< x \\
yz &= x \\
\textbf{Contradiction} \\
\end{align*}

However, I was unsatisfied with this, because while it shows clearly that $y$ and $z$ cannot both be greater or less than $\sqrt{x}$, it fails to show that one must be greater and the other less, and Proof by Lack of Other Options doesn't seem like a very good way of going about things.  

At this point I thought about how I could express a product in a different way such that the assertion I found in the book would become more apparent.  What I came up was this:

\begin{align*}
x &= y \times z \\
x &= (\sqrt{x} \times \frac{i}{j}) \times (\sqrt{x} \times \frac{j}{i})
\end{align*}

This definition is rather nice for a number of reasons.  For one, it handles the case where $y$ and $z$ are equal, whereas most of my previous scratch work handled these cases separately.  For another, it defines $x$ as a product of it's own square root, which allows us to clearly show the relationship between the two product roots $y$ and $z$ in such a way that the greater than / less than relationship is clear. Because the two fractions involving $i$ and $j$ are inverses, one side of the product always rises as the other falls. The two fractions involving $i$ and $j$ cancel each other out, leaving the multiplication two square roots, which is trivial.

So, the assertion in the book is correct, shown clearly by defining a product with square roots and two inverted fractions as above.

\section{Limitations}

I believe this only works for values of X that are positive and greater than 1.  I'm very confident that it only works for positive numbers, but the greater than 1 restriction is only a hunch.

\end{document}
